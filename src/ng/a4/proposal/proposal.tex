\documentclass{article}

\usepackage[english]{babel}
\usepackage{amsmath}
\usepackage{amsfonts}
\usepackage{fancyhdr}
\usepackage{tabularx}
\usepackage{listings}
\usepackage{parskip}
\usepackage{enumitem}
\usepackage{hhline}
\usepackage{algorithm2e}
\usepackage{geometry}
\usepackage{mathtools}
\usepackage{multicol}
\usepackage{array}

\pagestyle{fancy}

\lhead{Nicolas Guillemot}
\rhead{CSC 486a Assignment 4 Proposal}

\begin{document}

\title{Topic Proposal for Assignment 4}
\date{June 6, 2014}
\author{Nicolas Guillemot}
\maketitle

\section*{Topic}

I want to learn more about skeletal animation. Therefore, I will try to implement the md5 skeletal animation format. Md5 is an industry-standard format used by many games, so it would be good for me to become familiar with it. I've also been designing a rendering engine while doing homework for this class, and this would be a useful addition to it.

\section*{The Work}

There are many resources on the web that explain the details of the md5 skeletal animation format spec. Implementing it requires four steps:

\begin{itemize}
    \item Parse the .md5 model file.
    \item Interpolate the joints at every frame.
    \item Interpolate the orientation of the quaternions at every frame.
    \item Pass the bone information to the GPU to render the mesh.
\end{itemize}

\end{document}
